\chapter{TEST PLAN (TP) for Game Rules and Play}

{\centering\selectlanguage{english}\bfseries\color{black}
Version 1.1, October 2013
\par}

\ 

{\centering\selectlanguage{english}\bfseries\color{black}
FOREWORD
\par}

{\selectlanguage{english}\color{black}
This template was created to provide system and software development
projects with a model Test Plan (TP) document template.
The template is based on IEEE 829 Format.
It has been edited and updated by
Dr. Clint Jeffery for use in UI CS 383.}

{\selectlanguage{english}\color{black}
The TP template begins on the next page. \ Just throw away this page
and enter your project specifications into the following template.
\ Don{\textquoteright}t forget to change the headers and footers as
necessary. \ The following conventions are used to guide you in
developing your TP:}

{\selectlanguage{english}\color{black}
\foreignlanguage{english}{[ Text
]\ \ }\foreignlanguage{english}{\textbf{Replace}}\foreignlanguage{english}{
this text with your project design text.}}

{\selectlanguage{english}\color{black}
\foreignlanguage{english}{\textit{\ }}\foreignlanguage{english}{\textit{text
in italics }}\foreignlanguage{english}{\ \ Notes/instructions to the
author. }\foreignlanguage{english}{\textbf{Delete in your finished
document.}}}


\bigskip

{\centering\bfseries\color{black}
TEST PLAN (TP)}

{\centering\selectlanguage{english}\bfseries\color{black}
FOR
\par}


\bigskip

{\centering\selectlanguage{english}\bfseries\color{black}
TEST Plan for Spells and Characters
\par}


\bigskip


\bigskip


\bigskip

\begin{figure}
\centering
%\includegraphics[width=3.4354in,height=0.6126in]{uislogan.png}
\end{figure}

\bigskip


\bigskip


\bigskip


\bigskip

{\centering\selectlanguage{english}\bfseries\color{black}
Version 1.1.0
\par}

{\centering\selectlanguage{english}\bfseries\color{black}
May 1st, 2014
\par}


\bigskip


\bigskip

{\centering\selectlanguage{english}\bfseries\color{black}
Prepared for:
\par}

{\centering\selectlanguage{english}\bfseries\color{black}
CS 383 Course Project
\par}


\bigskip


\bigskip

{\centering\selectlanguage{english}\bfseries\color{black}
Prepared by:
\par}

{\centering\selectlanguage{english}\bfseries\color{black}
Tao Zhang, Cameron Simon, Matthrew Brown, Tyler Jaszkowiak, Ian Westrope, ChiHsiang Wang (Rule Team)
\par}

{\centering\selectlanguage{english}\bfseries\color{black}
University of Idaho
\par}

{\centering\selectlanguage{english}\bfseries\color{black}
Moscow, ID \ 83844-1010
\par}

{\centering\selectlanguage{english}\bfseries\color{black}
CS383 TPD
\par}

\pagebreak

{\centering\selectlanguage{english}\bfseries\color{black}
RECORD OF CHANGES (Change History)
\par}

\begin{flushleft}
\tablehead{}
\begin{supertabular}{|m{0.5462598in}|m{0.6712598in}|m{1.4212599in}|m{0.23375985in}|m{1.7962599in}|m{0.7337598in}|m{0.6295598in}|}
\hline
~

\centering {\selectlanguage{english}\bfseries\color{black} Change}\par

\centering {\selectlanguage{english}\bfseries\color{black} Number}\par

~
 &
~

\centering \selectlanguage{english}\bfseries\color{black} Date completed
&
~

\centering {\selectlanguage{english}\bfseries\color{black} Location of
change }\par

\centering \selectlanguage{english}\bfseries\color{black} (e.g., page or
figure \#) &
~

\centering {\selectlanguage{english}\bfseries\color{black} A}\par

\centering \selectlanguage{english}\bfseries\color{black} M\newline
D  &
~

\centering {\selectlanguage{english}\bfseries\color{black} Brief
description }\par

\centering \selectlanguage{english}\bfseries\color{black} of change &
~

\centering \selectlanguage{english}\bfseries\color{black} Approved by
(initials) &
~

\centering {\bfseries\color{black} Date }\par

\centering\arraybslash\bfseries\color{black}
approved\\

 &

 &

 &

 &

 &

 &

\\\hline
~
 &
~
 &
~
 &
~
 &
~
 &
~
 &
~
\\\hline
~
 &
~
 &
~
 &
~
 &
~
 &
~
 &
~
\\\hline
~
 &
~
 &
~
 &
~
 &
~
 &
~
 &
~
\\\hline
~
 &
~
 &
~
 &
~
 &
~
 &
~
 &
~
\\\hline
~
 &
~
 &
~
 &
~
 &
~
 &
~
 &
~
\\\hline
~
 &
~
 &
~
 &
~
 &
~
 &
~
 &
~
\\\hline
~
 &
~
 &
~
 &
~
 &
~
 &
~
 &
~
\\\hline
~
 &
~
 &
~
 &
~
 &
~
 &
~
 &
~
\\\hline
~
 &
~
 &
~
 &
~
 &
~
 &
~
 &
~
\\\hline
~
 &
~
 &
~
 &
~
 &
~
 &
~
 &
~
\\\hline
~
 &
~
 &
~
 &
~
 &
~
 &
~
 &
~
\\\hline
~
 &
~
 &
~
 &
~
 &
~
 &
~
 &
~
\\\hline
~
 &
~
 &
~
 &
~
 &
~
 &
~
 &
~
\\\hline
~
 &
~
 &
~
 &
~
 &
~
 &
~
 &
~
\\\hline
~
 &
~
 &
~
 &
~
 &
~
 &
~
 &
~
\\\hline
~
 &
~
 &
~
 &
~
 &
~
 &
~
 &
~
\\\hline
~
 &
~
 &
~
 &
~
 &
~
 &
~
 &
~
\\\hline
~
 &
~
 &
~
 &
~
 &
~
 &
~
 &
~
\\\hline
~
 &
~
 &
~
 &
~
 &
~
 &
~
 &
~
\\\hline
~
 &
~
 &
~
 &
~
 &
~
 &
~
 &
~
\\\hline
~
 &
~
 &
~
 &
~
 &
~
 &
~
 &
~
\\\hline
\end{supertabular}
\end{flushleft}
{\selectlanguage{english}\color{black}
A - ADDED \ M - MODIFIED \ D -- DELETED}

{\centering\selectlanguage{english}\bfseries\color{black}
[ put program /system name here ]
\par}

\pagebreak

{\centering\selectlanguage{english}\bfseries\color{black}
TABLE OF CONTENTS
\par}

{\selectlanguage{english}\bfseries\color{black}
Section\ \ Page}

\setcounter{tocdepth}{9}
\renewcommand\contentsname{}
\tableofcontents

\bigskip

\bigskip
\setcounter{page}{1}\pagestyle{Convertiv}

\section[IDENTIFIER]{\selectlanguage{english}\bfseries\color{black}
TEST PLAN IDENTIFIER}

{\selectlanguage{english}\color{black}
SWORDS \& SORCERY TEST PLAN VERSION 1.0, CS 383 TEAM

This test plan has no unique identifier other than Spring 2014 CS 383 sworsorc. This particular test plan is in conjunction to the rules and game play group test plans. There are many test plans for this group do to the size and span of the group's contents. This test plan, for example, will be known as the Units test plan.}
%{\selectlanguage{english}\color{black}
%[Insert text here.]}


\section[REFERENCES]{\selectlanguage{english}\bfseries\color{black}
REFERENCES}
%{\selectlanguage{english}\itshape\color{black}

{\selectlanguage{english}\color{black}
Use Cases and State diagrams are available at https://github.com/cjeffery/sworsorc/tree/master/doc }



\section[INTRODUCTION]{\bfseries\color{black} INTRODUCTION}

{\selectlanguage{english}\color{black}

The purpose of this test plan is to state the processes used by Game Rules and Play Team in testing the Spells and Characters for the Swords \& Sorcery project.

}

\section[TEST ITEMS]{\bfseries\color{black} TEST ITEMS}

%{\selectlanguage{english}\itshape\color{black}
%These are things you intend to test within the scope of this test
%plan. Essentially, something you will test, a list of what is to be
%tested. This can be developed from the software application
%inventories as well as other sources of documentation and information.
%}

{\selectlanguage{english}\color{black}
\begin{itemize}
\item Class Interfaces
\item Class Interactions
\item Spells Implementation
\item Character 
\end{itemize}
}

\section[SOFTWARE RISK ISSUES]{\bfseries\color{black} SOFTWARE RISK ISSUES}
{\selectlanguage{english}\color{black}

Testing will mainly focus on code written by Game Rules and Play Team members, more specifically spells and characters. This plan largely focuses on providing a testing plan that adequately covers both rules and implementation.
As the code base is not yet complete, the complexity of the code cannot be yet determined. Also, since the documentation is often incomplete, risk also arises because of poor documentation. Complete and improved documentation of all areas of code with reduce this risk. Since our specific code for spells and characters is not dependent on the third party software, there is a little risk comes from new version of software or failure of any third parties. 

%\begin{itemize}
%\item   1. Delivery of a third party product.
%\item   2. New version of interfacing software
%\item   3. Ability to use and understand a new package/tool, etc.
%\item   4. Extremely complex functions
%\item   5. Modifications to components with a past history of failure
%\item   6. Poorly documented modules or change requests 
%\end{itemize}
}
%{\selectlanguage{english}\color{black}
%[Insert text here.]}

\section[FEATURES TO BE TESTED]{\bfseries\color{black} FEATURES TO BE TESTED}
{\selectlanguage{english}\color{black}

\begin{itemize}
\item Select character on GUI
\item Select Spell
\item Effects of Partial Spells
\item Manna Costing
\end{itemize}

}
%{\selectlanguage{english}\color{black}
%[Insert text here.]}

\section[FEATURES NOT TO BE TESTED]{\bfseries\color{black}
	 FEATURES NOT TO BE TESTED}
%{\selectlanguage{english}\itshape\color{black}

%This is a listing of what is NOT to be tested from both the Users
%viewpoint of what the system does and a configuration
%management/version control view. This is not a technical description
%of the software, but a USERS view of the functions.

}
{\selectlanguage{english}\color{black}
\begin{itemize}
\item Speed of software
\end{itemize}

\section[APPROACH]{\bfseries\color{black} APPROACH}
%{\selectlanguage{english}\itshape\color{black}

%This is your overall test strategy for this test plan; it should be
%appropriate to the level of the plan (master, acceptance, etc.) and
%should be in agreement with all higher and lower levels of
%plans. Overall rules and processes should be identified. 

%\begin{itemize}
%\item Are any special tools to be used? What are they?
%\item What metrics will be collected for this test?
%\item How many configurations/platforms are to be tested?
%\item How will elements in the design deemed "untestable" be processed?
%\end{itemize}
%}
{\selectlanguage{english}\color{black}
\subsection{Testing Tools}
JUnit on NetBeans.

\subsection{Metrics}

\subsection{Configurations}

\subsection{Software}
The software will be developed with Java 1.8.

\subsection{Hardware}

\subsection{Automated Testing}
The automated test process will be Unit testing ...
\subsubsection{Unit Testing}
Not yet.
%Unit testing will be used to validate individual classes objects. Some objects will not be required to be validated by unit testing, such as the user interface and the supporting classes and any classes or code that is exempted according to section 5.\\
%Each team should develop TCSs as needed for their code. Each TCS should include tests for all classes as required and for each test, validate as many inputs as needed to test that the class operates as required by the rules covered in the SSRS, and also according to an requirements and design documents applicable to that class.

\subsection{Manual Testing}
Manual testing will be required for the portions of the program that can not undergo  automated testing. This section applies to the testing of the user interface to simulate user orientated testing to verify conformance to the documented use cases.
}

\section[ITEM PASS/FAIL CRITERIA]{\bfseries\color{black}
	 ITEM PASS/FAIL CRITERIA}
%{\selectlanguage{english}\itshape\color{black}
%What are the Completion criteria for this plan? This is a critical
%aspect of any test plan and should be appropriate to the level of the plan.
%}
{\selectlanguage{english}\color{black}
\subsection{Reporting a failure}
If a failure happens during the execution of any test, a failure report should be submitted to provide information of the failure. Failure should be reported in the issue page of the repository.
\newline
The name of the issue should be the name of the name of the code file and the name of the test if applicable.

\subsubsection{What Constitutes failure?}
And deviation from a specification, e.g. SRS, UML diagrams.

\subsubsection{What Does Not Constitute a failure?}
Any unit or action that does not have any requirements documentation, cannot cause a failure.

\subsubsection{What to do when a failure is discovered?}
Produce a SCR to document each failure that needs to be corrected.

\subsubsection{What if a specification document is incorrect (e.g. outdated, misstated)?}
This also constitutes a failure and an SCR should be created.

\subsubsection{What sections to include in an SCR?}
Failure Identified 
\newline
Expected Outcome 
\newline
Actual Behavior 
\newline
Steps to reproduce
\newline
Each section should be brief and to the point, but yet convey enough information for the coder.

\subsection{Unit Testing Pass/Fail Criteria}

\subsection{Intergration Testing Pass/Fail Criteria}
\subsubsection{Character}
\begin{center}
\begin{supertabular}{|p{4.5cm}|p{4.5cm}|p{4.5cm}|}
\hline
    \textbf{Rule Description}
    &
    \textbf{Test Description}
    &
    \textbf{Expected Result}
\\\hline
    TBD
    &
    createCharacter function is called with 
    specific character name from CharacterMaker.java.
    &
    A new character object is created and returned 
    that contains all of the specified characters information. 
\\\hline    
\end{supertabular}
\end{center}

\subsubsection{Spells}
\begin{center}
\begin{supertabular}{|p{4.5cm}|p{4.5cm}|p{4.5cm}|}
\hline
    \textbf{Rule Description}
    &
    \textbf{Test Description}
    &
    \textbf{Expected Result}
\\\hline
    Character with Power Level should have a spell book.
    &
    Generate a spell book for the character.
    &
    A frame with a list of spells should be shown on the screen.
\\\hline 
    Show spell description.
    &
    Click on the Spell button.
    &
    A frame will be displayed with all information about the spell.
\\\hline   
    A target need to be selected for most of the spells.
    &
    Click on cast button on the frame of displaying information about the spell. 
    Then select a target by right click the target on the map.
    &
    A target is selected to cast the spell.
\\\hline
\end{supertabular}
\end{center}





}

\section[SUSPENSION CRITERIA]{\bfseries\color{black}
	 SUSPENSION CRITERIA AND RESUMPTION REQUIREMENTS}
{\selectlanguage{english}\itshape\color{black}
If the number or type of defects reaches a point where the follow on
testing has no value, it makes no sense to continue the test; you are
just wasting resources.

Specify what constitutes stoppage for a test or series of tests and
what is the acceptable level of defects that will allow the testing to
proceed past the defects. 
}
{\selectlanguage{english}\color{black}
[Insert text here.]}

\section[TEST DELIVERABLES]{\bfseries\color{black} TEST DELIVERABLES}
{\selectlanguage{english}\itshape\color{black}
What is to be delivered as part of this plan?

\begin{itemize}
\item Test plan document
\item Test cases
\item Relevant error logs or problem reports
\end{itemize}
One thing that is not a test deliverable is the software itself that
is listed under test items and is delivered by development.
}
{\selectlanguage{english}\color{black}
[Insert text here.]}

\section[REMAINING TEST TASKS]{\bfseries\color{black} REMAINING TEST TASKS}
{\selectlanguage{english}\itshape\color{black}
If this is a multi-phase process or if the application is to be
released in increments there may be parts of the application that this
plan does not address. These areas need to be identified to avoid any
confusion should defects be reported back on those future
functions. This will also allow the users and testers to avoid
incomplete functions and prevent waste of resources chasing non-defects.
}
{\selectlanguage{english}\color{black}
[Insert text here.]}

\section[ENVIRONMENTAL NEEDS]{\bfseries\color{black} ENVIRONMENTAL NEEDS}
{\selectlanguage{english}\itshape\color{black}
Are there any special requirements for this test plan, such as:

\begin{itemize}
\item Special hardware such as simulators, static generators etc.
\item How will test data be provided. Are there special collection
	requirements or specific ranges of data that must be provided? 
\end{itemize}

}
{\selectlanguage{english}\color{black}
[Insert text here.]}

\section[STAFFING AND TRAINING NEEDS]{\bfseries\color{black}
	 STAFFING AND TRAINING NEEDS}
{\selectlanguage{english}\itshape\color{black}
Training on the application/system.

Training for any test tools to be used. 
}
{\selectlanguage{english}\color{black}
[Insert text here.]}

\section[RESPONSIBILITIES]{\bfseries\color{black} RESPONSIBILITIES}
{\selectlanguage{english}\itshape\color{black}
Who is in charge?

This issue includes all areas of the plan. Here are some examples:

\begin{itemize}
\item Selecting features to be tested and not tested.
\item Ensuring all required elements are in place for testing. 
\end{itemize}
}
{\selectlanguage{english}\color{black}
[Insert text here.]}

\section[SCHEDULE]{\bfseries\color{black} SCHEDULE}
{\selectlanguage{english}\itshape\color{black}

Should be based on realistic and validated estimates. If the estimates
for the development of the application are inaccurate, the entire
project plan will slip and the testing is part of the overall project plan.

}
{\selectlanguage{english}\color{black}
[Insert text here.]}

\section[PLANNING RISKS AND CONTINGENCIES]{\bfseries\color{black}
	 PLANNING RISKS AND CONTINGENCIES}
{\selectlanguage{english}\itshape\color{black}

What are the overall risks to the project with an emphasis on the
testing process? Specify what will be done for various risk events.

}
{\selectlanguage{english}\color{black}
[Insert text here.]}

\section[APPROVALS]{\bfseries\color{black} APPROVALS}
{\selectlanguage{english}\itshape\color{black}

Who can approve the process as complete and allow the project to
proceed to the next level (depending on the level of the plan)? 

}
{\selectlanguage{english}\color{black}
[Insert text here.]}

\section[GLOSSARY]{\bfseries\color{black} GLOSSARY}

{\selectlanguage{english}\itshape\color{black}

Used to define terms and acronyms used in the document, and testing in
general, to eliminate confusion and promote consistent communications.

}
{\selectlanguage{english}\color{black}
[Insert text here.]}



\clearpage\setcounter{page}{1}\pagestyle{Convertviii}
\section[APPENDIX A. \ [insert name
here{]}]{\selectlanguage{english}\bfseries\color{black} APPENDIX A.
\ [insert name here]}
{\selectlanguage{english}\itshape\color{black}
Include copies of test examples, etc. supplied or
derived from the customer. \ Appendices are labeled A, B, {\dots}n.
\ \ Reference each appendix as appropriate in the text of the document.
}

{\selectlanguage{english}\color{black}
\ [ insert appendix A here ]}

\clearpage\setcounter{page}{1}\pagestyle{Convertix}
\section[APPENDIX B. \ [insert name
here{]}]{\selectlanguage{english}\bfseries\color{black} APPENDIX B.
\ [insert name here]}

\bigskip

{\selectlanguage{english}\color{black}
[ insert appendix B here ]}


\bigskip
