\chapter{TEST PLAN (TP) TEMPLATE}

{\centering\selectlanguage{english}\bfseries\color{black}
Version 1.1, October 2013
\par}

\ 

{\centering\selectlanguage{english}\bfseries\color{black}
FOREWORD
\par}

{\selectlanguage{english}\color{black}
This template was created to provide system and software development
projects with a model Test Plan (TP) document template.
The template is based on IEEE 829 Format.
It has been edited and updated by
Dr. Clint Jeffery for use in UI CS 383.}

{\selectlanguage{english}\color{black}
The TP template begins on the next page. \ Just throw away this page
and enter your project specifications into the following template.
\ Don{\textquoteright}t forget to change the headers and footers as
necessary. \ The following conventions are used to guide you in
developing your TP:}

{\selectlanguage{english}\color{black}
\foreignlanguage{english}{[ Text
]\ \ }\foreignlanguage{english}{\textbf{Replace}}\foreignlanguage{english}{
this text with your project design text.}}

{\selectlanguage{english}\color{black}
\foreignlanguage{english}{\textit{\ }}\foreignlanguage{english}{\textit{text
in italics }}\foreignlanguage{english}{\ \ Notes/instructions to the
author. }\foreignlanguage{english}{\textbf{Delete in your finished
document.}}}


\bigskip

{\centering\bfseries\color{black}
TEST PLAN (TP)}

{\centering\selectlanguage{english}\bfseries\color{black}
FOR
\par}


\bigskip

{\centering\selectlanguage{english}\bfseries\color{black}
[ state program/system name here ]
\par}


\bigskip


\bigskip


\bigskip

\begin{figure}
\centering
\includegraphics[width=3.4354in,height=0.6126in]{uislogan.png}
\end{figure}

\bigskip


\bigskip


\bigskip


\bigskip

{\centering\selectlanguage{english}\bfseries\color{black}
Version [[insert version number]]
\par}

{\centering\selectlanguage{english}\bfseries\color{black}
[[insert date]]
\par}


\bigskip


\bigskip

{\centering\selectlanguage{english}\bfseries\color{black}
Prepared for:
\par}

{\centering\selectlanguage{english}\bfseries\color{black}
[ state customer name(s) here ]
\par}


\bigskip


\bigskip

{\centering\selectlanguage{english}\bfseries\color{black}
Prepared by:
\par}

{\centering\selectlanguage{english}\bfseries\color{black}
[insert your name(s)]
\par}

{\centering\selectlanguage{english}\bfseries\color{black}
University of Idaho
\par}

{\centering\selectlanguage{english}\bfseries\color{black}
Moscow, ID \ 83844-1010
\par}

{\centering\selectlanguage{english}\bfseries\color{black}
CS383 TPD
\par}

\pagebreak

{\centering\selectlanguage{english}\bfseries\color{black}
RECORD OF CHANGES (Change History)
\par}

\begin{flushleft}
\tablehead{}
\begin{supertabular}{|m{0.5462598in}|m{0.6712598in}|m{1.4212599in}|m{0.23375985in}|m{1.7962599in}|m{0.7337598in}|m{0.6295598in}|}
\hline
~

\centering {\selectlanguage{english}\bfseries\color{black} Change}\par

\centering {\selectlanguage{english}\bfseries\color{black} Number}\par

~
 &
~

\centering \selectlanguage{english}\bfseries\color{black} Date completed
&
~

\centering {\selectlanguage{english}\bfseries\color{black} Location of
change }\par

\centering \selectlanguage{english}\bfseries\color{black} (e.g., page or
figure \#) &
~

\centering {\selectlanguage{english}\bfseries\color{black} A}\par

\centering \selectlanguage{english}\bfseries\color{black} M\newline
D  &
~

\centering {\selectlanguage{english}\bfseries\color{black} Brief
description }\par

\centering \selectlanguage{english}\bfseries\color{black} of change &
~

\centering \selectlanguage{english}\bfseries\color{black} Approved by
(initials) &
~

\centering {\bfseries\color{black} Date }\par

\centering\arraybslash\bfseries\color{black}
approved\\

 &

 &

 &

 &

 &

 &

\\\hline
~
 &
~
 &
~
 &
~
 &
~
 &
~
 &
~
\\\hline
~
 &
~
 &
~
 &
~
 &
~
 &
~
 &
~
\\\hline
~
 &
~
 &
~
 &
~
 &
~
 &
~
 &
~
\\\hline
~
 &
~
 &
~
 &
~
 &
~
 &
~
 &
~
\\\hline
~
 &
~
 &
~
 &
~
 &
~
 &
~
 &
~
\\\hline
~
 &
~
 &
~
 &
~
 &
~
 &
~
 &
~
\\\hline
~
 &
~
 &
~
 &
~
 &
~
 &
~
 &
~
\\\hline
~
 &
~
 &
~
 &
~
 &
~
 &
~
 &
~
\\\hline
~
 &
~
 &
~
 &
~
 &
~
 &
~
 &
~
\\\hline
~
 &
~
 &
~
 &
~
 &
~
 &
~
 &
~
\\\hline
~
 &
~
 &
~
 &
~
 &
~
 &
~
 &
~
\\\hline
~
 &
~
 &
~
 &
~
 &
~
 &
~
 &
~
\\\hline
~
 &
~
 &
~
 &
~
 &
~
 &
~
 &
~
\\\hline
~
 &
~
 &
~
 &
~
 &
~
 &
~
 &
~
\\\hline
~
 &
~
 &
~
 &
~
 &
~
 &
~
 &
~
\\\hline
~
 &
~
 &
~
 &
~
 &
~
 &
~
 &
~
\\\hline
~
 &
~
 &
~
 &
~
 &
~
 &
~
 &
~
\\\hline
~
 &
~
 &
~
 &
~
 &
~
 &
~
 &
~
\\\hline
~
 &
~
 &
~
 &
~
 &
~
 &
~
 &
~
\\\hline
~
 &
~
 &
~
 &
~
 &
~
 &
~
 &
~
\\\hline
~
 &
~
 &
~
 &
~
 &
~
 &
~
 &
~
\\\hline
\end{supertabular}
\end{flushleft}
{\selectlanguage{english}\color{black}
A - ADDED \ M - MODIFIED \ D -- DELETED}

{\centering\selectlanguage{english}\bfseries\color{black}
[ put program /system name here ]
\par}

\pagebreak

{\centering\selectlanguage{english}\bfseries\color{black}
TABLE OF CONTENTS
\par}

{\selectlanguage{english}\bfseries\color{black}
Section\ \ Page}

\setcounter{tocdepth}{9}
\renewcommand\contentsname{}
\tableofcontents

\bigskip

\bigskip
\setcounter{page}{1}\pagestyle{Convertiv}

\section[IDENTIFIER]{\selectlanguage{english}\bfseries\color{black}
TEST PLAN IDENTIFIER}

{\selectlanguage{english}\itshape\color{black}
Some type of unique company generated number to identify this test
plan, its level and the level of software that it is related
to. Preferably the test plan level will be the same as the related
software level. The number may also identify whether the test plan is
a Master plan, a Level plan, an integration plan or whichever plan
level it represents. This is to assist in coordinating software and
testware versions within configuration management.
}

{\selectlanguage{english}\color{black}
[Insert text here.]}


\section[REFERENCES]{\selectlanguage{english}\bfseries\color{black}
REFERENCES}

{\selectlanguage{english}\itshape\color{black}
List all documents that support this test plan. Refer to the actual
version/release number of the document as stored in the configuration
management system. Do not duplicate the text from other documents as
this will reduce the viability of this document and increase the
maintenance effort.
}

{\selectlanguage{english}\color{black}
[Insert text here.]}



\section[INTRODUCTION]{\bfseries\color{black} INTRODUCTION}

{\selectlanguage{english}\itshape\color{black}

State the purpose of the Plan, possibly identifying the level of the
plan (master etc.). This is essentially the executive summary part of
the plan. 

}

\section[TEST ITEMS]{\bfseries\color{black} TEST ITEMS}

{\selectlanguage{english}\itshape\color{black}
These are things you intend to test within the scope of this test
plan. Essentially, something you will test, a list of what is to be
tested. This can be developed from the software application
inventories as well as other sources of documentation and information.
}

{\selectlanguage{english}\color{black}
[Insert text here.]}

\section[SOFTWARE RISK ISSUES]{\bfseries\color{black} SOFTWARE RISK ISSUES}
{\selectlanguage{english}\itshape\color{black}

Identify what software is to be tested and what the critical areas
are, such as:

\begin{itemize}
\item   1. Delivery of a third party product.
\item   2. New version of interfacing software
\item   3. Ability to use and understand a new package/tool, etc.
\item   4. Extremely complex functions
\item   5. Modifications to components with a past history of failure
\item   6. Poorly documented modules or change requests 
\end{itemize}
}
{\selectlanguage{english}\color{black}
[Insert text here.]}

\section[FEATURES TO BE TESTED]{\bfseries\color{black} FEATURES TO BE TESTED}
{\selectlanguage{english}\itshape\color{black}

This is a listing of what is to be tested from the USERS viewpoint of
what the system does. This is not a technical description of the
software, but a USERS view of the functions.

}
{\selectlanguage{english}\color{black}
[Insert text here.]}

\section[FEATURES NOT TO BE TESTED]{\bfseries\color{black}
	 FEATURES NOT TO BE TESTED}
{\selectlanguage{english}\itshape\color{black}

This is a listing of what is NOT to be tested from both the Users
viewpoint of what the system does and a configuration
management/version control view. This is not a technical description
of the software, but a USERS view of the functions.

}
{\selectlanguage{english}\color{black}
[Insert text here.]}

\section[APPROACH]{\bfseries\color{black} APPROACH}
{\selectlanguage{english}\itshape\color{black}

This is your overall test strategy for this test plan; it should be
appropriate to the level of the plan (master, acceptance, etc.) and
should be in agreement with all higher and lower levels of
plans. Overall rules and processes should be identified. 

\begin{itemize}
\item Are any special tools to be used? What are they?
\item What metrics will be collected for this test?
\item How many configurations/platforms are to be tested?
\item How will elements in the design deemed "untestable" be processed?
\end{itemize}
}
{\selectlanguage{english}\color{black}
[Insert text here.]}

\section[ITEM PASS/FAIL CRITERIA]{\bfseries\color{black}
	 ITEM PASS/FAIL CRITERIA}
{\selectlanguage{english}\itshape\color{black}
What are the Completion criteria for this plan? This is a critical
aspect of any test plan and should be appropriate to the level of the plan.
}
{\selectlanguage{english}\color{black}
[Insert text here.]}

\section[SUSPENSION CRITERIA]{\bfseries\color{black}
	 SUSPENSION CRITERIA AND RESUMPTION REQUIREMENTS}
{\selectlanguage{english}\itshape\color{black}
If the number or type of defects reaches a point where the follow on
testing has no value, it makes no sense to continue the test; you are
just wasting resources.

Specify what constitutes stoppage for a test or series of tests and
what is the acceptable level of defects that will allow the testing to
proceed past the defects. 
}
{\selectlanguage{english}\color{black}
[Insert text here.]}

\section[TEST DELIVERABLES]{\bfseries\color{black} TEST DELIVERABLES}
{\selectlanguage{english}\itshape\color{black}
What is to be delivered as part of this plan?

\begin{itemize}
\item Test plan document
\item Test cases
\item Relevant error logs or problem reports
\end{itemize}
One thing that is not a test deliverable is the software itself that
is listed under test items and is delivered by development.
}
{\selectlanguage{english}\color{black}
[Insert text here.]}

\section[REMAINING TEST TASKS]{\bfseries\color{black} REMAINING TEST TASKS}
{\selectlanguage{english}\itshape\color{black}
If this is a multi-phase process or if the application is to be
released in increments there may be parts of the application that this
plan does not address. These areas need to be identified to avoid any
confusion should defects be reported back on those future
functions. This will also allow the users and testers to avoid
incomplete functions and prevent waste of resources chasing non-defects.
}
{\selectlanguage{english}\color{black}
[Insert text here.]}

\section[ENVIRONMENTAL NEEDS]{\bfseries\color{black} ENVIRONMENTAL NEEDS}
{\selectlanguage{english}\itshape\color{black}
Are there any special requirements for this test plan, such as:

\begin{itemize}
\item Special hardware such as simulators, static generators etc.
\item How will test data be provided. Are there special collection
	requirements or specific ranges of data that must be provided? 
\end{itemize}

}
{\selectlanguage{english}\color{black}
[Insert text here.]}

\section[STAFFING AND TRAINING NEEDS]{\bfseries\color{black}
	 STAFFING AND TRAINING NEEDS}
{\selectlanguage{english}\itshape\color{black}
Training on the application/system.

Training for any test tools to be used. 
}
{\selectlanguage{english}\color{black}
[Insert text here.]}

\section[RESPONSIBILITIES]{\bfseries\color{black} RESPONSIBILITIES}
{\selectlanguage{english}\itshape\color{black}
Who is in charge?

This issue includes all areas of the plan. Here are some examples:

\begin{itemize}
\item Selecting features to be tested and not tested.
\item Ensuring all required elements are in place for testing. 
\end{itemize}
}
{\selectlanguage{english}\color{black}
[Insert text here.]}

\section[SCHEDULE]{\bfseries\color{black} SCHEDULE}
{\selectlanguage{english}\itshape\color{black}

Should be based on realistic and validated estimates. If the estimates
for the development of the application are inaccurate, the entire
project plan will slip and the testing is part of the overall project plan.

}
{\selectlanguage{english}\color{black}
[Insert text here.]}

\section[PLANNING RISKS AND CONTINGENCIES]{\bfseries\color{black}
	 PLANNING RISKS AND CONTINGENCIES}
{\selectlanguage{english}\itshape\color{black}

What are the overall risks to the project with an emphasis on the
testing process? Specify what will be done for various risk events.

}
{\selectlanguage{english}\color{black}
[Insert text here.]}

\section[APPROVALS]{\bfseries\color{black} APPROVALS}
{\selectlanguage{english}\itshape\color{black}

Who can approve the process as complete and allow the project to
proceed to the next level (depending on the level of the plan)? 

}
{\selectlanguage{english}\color{black}
[Insert text here.]}

\section[GLOSSARY]{\bfseries\color{black} GLOSSARY}

{\selectlanguage{english}\itshape\color{black}

Used to define terms and acronyms used in the document, and testing in
general, to eliminate confusion and promote consistent communications.

}
{\selectlanguage{english}\color{black}
[Insert text here.]}



\clearpage\setcounter{page}{1}\pagestyle{Convertviii}
\section[APPENDIX A. \ [insert name
here{]}]{\selectlanguage{english}\bfseries\color{black} APPENDIX A.
\ [insert name here]}
{\selectlanguage{english}\itshape\color{black}
Include copies of test examples, etc. supplied or
derived from the customer. \ Appendices are labeled A, B, {\dots}n.
\ \ Reference each appendix as appropriate in the text of the document.
}

{\selectlanguage{english}\color{black}
\ [ insert appendix A here ]}

\clearpage\setcounter{page}{1}\pagestyle{Convertix}
\section[APPENDIX B. \ [insert name
here{]}]{\selectlanguage{english}\bfseries\color{black} APPENDIX B.
\ [insert name here]}

\bigskip

{\selectlanguage{english}\color{black}
[ insert appendix B here ]}


\bigskip
