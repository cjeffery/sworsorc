\documentclass[12pt,a4paper]{article}
\usepackage[utf8]{inputenc}
\usepackage{amsmath}
\usepackage{amsfonts}
\usepackage{amssymb}
\usepackage{graphicx}
\usepackage{tabularx}
\parskip=0pt
\usepackage[left=1in,right=1in,top=1in,bottom=1in]{geometry}
\begin{document}
\begin{titlepage}
    \centering
    \vfill
    \vfill
    \includegraphics{UIGraphic}
    \vfill
    {\bfseries\Large
        Swords and Sorcery Implementation Description\\
        University of Idaho, CS 383, Spring 2014\\
        %\vskip2cm
        %Keith Drew, ... \\
        %\vskip2cm
        \today
    }    
    \vfill
\end{titlepage}

\section{Introduction and Document Description}
This is a description of the Swords and Sorcery implementation, both in general
and the implementation of major code modules.

This document particularly describes what was not covered in the design
documentation, as wekk as places where implementation differs from design.

Sections are:
\begin{itemize}
\item Implemetation Overview
\item Deployment, System Requirements, and Libraries
\item Building
\item Major Code Modules
\begin{itemize}
	\item The JavaFX HUD	
	\item The Map Rendering Code
	\item The Networking System
	\item The Map File Format
	\item The Movement Calculator
	\item ...
\end{itemize}
\end{itemize}

================================================================================
Implementation Overview:
================================================================================

Swords and Sorcery is programmed in Java 8 (after switching from Java 7),
and is cross platform across Windows, Mac and Linux.


It uses a 2D top-down view and graphics based on the physical game by SPI,
has a GUI programmed in JavaFX and swing, and some TCP networking
support.

Unfortunately the implementation remains incomplete, so while some basic
functionality exists, users will find it impossible to face eachother in battle
over the internet.

================================================================================
Deployment, System Requirements, and Libraries:
================================================================================

Swords and Sorcery requires the Java 8 JRE installed to run. This must be 
installed by the user independantly before trying to launch the game.

There is no installer package. The program takes the form of two .jar files
(one for the client, and one for the server), along with a library folder and a 
resources folder.

In Windows the user should be
able to launch the program by double clicking one of these .jar files. 

In other supported operating systems the user should be able to launch the program
with the following command line script

    java -jar client.jar

or 

    java -jar server.jar

The program must be launched from the command line in this manner if the user wishes
to view debug output.

The only required libraries are

\begin{itemize}
	\item json-simple 1.1.1
	\item controlsfx 8.0.5
\end{itemize}

.jar distributions of these two libraries are included.

Swords and Sorcery isn't the fastest Java code on the block, but should be
runnable by most contemporary systems.

================================================================================
Building:
================================================================================

Swords and Sorcery is hosted on github. If git is installed then the source can
be downloaded with the following command:

    git clone https://github.com/cjeffery/sworsorc.git

It was programmed using the Netbeans 8 IDE (after switching from netbeans 7.2)

Swords and Sorcery can be compiled from inside Netbeans 8 by selecting either
the <default config> or the Game build targets to build the client. The server
can be built by selecting the Server target.

Other build targets exist for debugging or legacy reasons and should be ignored.

Alternatively Swords and Sorcery can be built from the command line with ant,
using a build script similar to the following:

    ant -f build.xml -q -Dconfig=Game jar
    ant -f build.xml -q -Dconfig=Server jar

to build the client and server respectively.

Be aware that however the project is built, 
both client and server will build a file called sworsorc.jar

If both client and server have to exist side by side,
then the .jar files will have to be renamed to avoid overwriting eachother.

================================================================================
Major Code Modules
================================================================================

Swords and Sorcery has several major code modules. Amongst them the following
have interesting enough implementations to bear discussing:

\begin{itemize}
	\item The JavaFX HUD	
	\item The Map Rendering Code
	\item The Networking System
	\item The Map File Format
	\item The Movement Calculator
    \item ...
\end{itemize}

================================================================================
How Major Code Modules fit together
================================================================================
Not sure how feasible writing this section will be.
Ideally would be compared to design.

================================================================================
The JavaFX HUD
================================================================================
Describe Game.java and HUDController.java I guess

================================================================================
The Map Rendering Code
================================================================================
Colin will write this

================================================================================
The Networking System
================================================================================
Network team should write this

================================================================================
The Map File Format
================================================================================

================================================================================
The Movement Calculator
================================================================================
Be sure to mention that it uses a recursive algorithm

================================================================================
Gameplay Model Classes
================================================================================
Gameplay team knock yourself out.

UnitPool could also go here, or in it's own section.


\end{document}
