\chapter{TEST PLAN}


{\centering\selectlanguage{english}\bfseries\color{black}
Version 1.1\\
\today
\par}

\ 

{\centering\selectlanguage{english}\bfseries\color{black}
FOREWORD
\par}
\bigskip

{\centering\selectlanguage{english}\bfseries\color{black}
FOR
\par}


\bigskip

{\centering\selectlanguage{english}\bfseries\color{black}
Swords and Sorcery
\par}




\begin{figure}
\centering
%%\includegraphics[width=3.4354in,height=0.6126in]{uislogan.png}
\end{figure}

\bigskip


\bigskip

{\centering\selectlanguage{english}\bfseries\color{black}
Prepared for: 
\par
Software Engineering: CS 383\\
Dr. Jeffery
\par}

\bigskip


\bigskip

{\centering\selectlanguage{english}\bfseries\color{black}
Prepared by:
\par}

{\centering\selectlanguage{english}\bfseries\color{black}
Keith Drew,
 Ian Westrope,
 Jonathan Flake,
 Colin Clifford,
 David Klingenberg,
 Sean Shepherd,
 Chris,
 Gabe,
 John,
 Wayne, 
 Tao Zhang,
 Cameron Simon,
 Matthrew Brown,
 Tyler Jaszkowiak,
 Ian Westrope,
 ChiHsiang Wang
\par}

\bigskip

{\centering\selectlanguage{english}\bfseries\color{black}
University of Idaho
\par}

{\centering\selectlanguage{english}\bfseries\color{black}
Moscow, ID \ 83844-1010
\par}



\pagebreak

{\centering\selectlanguage{english}\bfseries\color{black}
RECORD OF CHANGES (Change History)
\par}

\begin{flushleft}
\tablehead{}
\begin{supertabular}{|m{0.5462598in}|m{0.6712598in}|m{1.4212599in}|m{0.23375985in}|m{1.7962599in}|m{0.7337598in}|m{0.6295598in}|}
\hline
~

\centering {\selectlanguage{english}\bfseries\color{black} Change}\par

\centering {\selectlanguage{english}\bfseries\color{black} Number}\par

~
 &
~

\centering \selectlanguage{english}\bfseries\color{black} Date completed
&
~

\centering {\selectlanguage{english}\bfseries\color{black} Location of
change }\par

\centering \selectlanguage{english}\bfseries\color{black} (e.g., page or
figure \#) &
~

\centering {\selectlanguage{english}\bfseries\color{black} A}\par

\centering \selectlanguage{english}\bfseries\color{black} M\newline
D  &
~

\centering {\selectlanguage{english}\bfseries\color{black} Brief
description }\par

\centering \selectlanguage{english}\bfseries\color{black} of change &
~

\centering \selectlanguage{english}\bfseries\color{black} Approved by
(initials) &
~

\centering {\bfseries\color{black} Date }\par

\centering\arraybslash\bfseries\color{black}
approved\\

 &

 &

 &

 &

 &

 &

\\\hline
~1
 &
~April, 2014
 &
~Original HUD team document.
 &
~A
 &
~Original HUD team document.
 &
~HUD Team
 &April, 2014
~
\\\hline
~2
 &
~May 5, 2014
 &
~Entire Document
 &
~M
 &
~Concatenated all team documents together.
 &
~DRK
 &
~May 5, 2014
\\\hline
~
 &
~
 &
~
 &
~
 &
~
 &
~
 &
~
\\\hline
~
 &
~
 &
~
 &
~
 &
~
 &
~
 &
~
\\\hline
~
 &
~
 &
~
 &
~
 &
~
 &
~
 &
~
\\\hline
~
 &
~
 &
~
 &
~
 &
~
 &
~
 &
~
\\\hline
~
 &
~
 &
~
 &
~
 &
~
 &
~
 &
~
\\\hline
~
 &
~
 &
~
 &
~
 &
~
 &
~
 &
~
\\\hline
~
 &
~
 &
~
 &
~
 &
~
 &
~
 &
~
\\\hline
~
 &
~
 &
~
 &
~
 &
~
 &
~
 &
~
\\\hline
~
 &
~
 &
~
 &
~
 &
~
 &
~
 &
~
\\\hline
~
 &
~
 &
~
 &
~
 &
~
 &
~
 &
~
\\\hline
~
 &
~
 &
~
 &
~
 &
~
 &
~
 &
~
\\\hline
~
 &
~
 &
~
 &
~
 &
~
 &
~
 &
~
\\\hline
~
 &
~
 &
~
 &
~
 &
~
 &
~
 &
~
\\\hline
~
 &
~
 &
~
 &
~
 &
~
 &
~
 &
~
\\\hline
~
 &
~
 &
~
 &
~
 &
~
 &
~
 &
~
\\\hline
~
 &
~
 &
~
 &
~
 &
~
 &
~
 &
~
\\\hline
~
 &
~
 &
~
 &
~
 &
~
 &
~
 &
~
\\\hline
~
 &
~
 &
~
 &
~
 &
~
 &
~
 &
~
\\\hline
~
 &
~
 &
~
 &
~
 &
~
 &
~
 &
~
\\\hline
\end{supertabular}
\end{flushleft}
{\selectlanguage{english}\color{black}
A - ADDED \ M - MODIFIED \ D -- DELETED}

{\centering\selectlanguage{english}\bfseries\color{black}
Swords and Sorcery Alpha 0.5
\par}

\pagebreak

{\centering\selectlanguage{english}\bfseries\color{black}
TABLE OF CONTENTS
\par}

{\selectlanguage{english}\bfseries\color{black}
Section\ \ Page}

\setcounter{tocdepth}{9}
\renewcommand\contentsname{}
\tableofcontents

\bigskip

\bigskip
\setcounter{page}{1}\pagestyle{Convertiv}

\section[IDENTIFIER]{\selectlanguage{english}\bfseries\color{black}
TEST PLAN IDENTIFIER}

{\selectlanguage{english}\color{black}
TestPlan 1.5\\}


\section[REFERENCES]{\selectlanguage{english}\bfseries\color{black}
REFERENCES}

{\selectlanguage{english}\color{black}
Use cases and UML diagrams are available on the class website, as well as game rules.\\
http://www2.cs.uidaho.edu/~jeffery/courses/383/ \\
The code itself can be found at https://github.com/cjeffery/sworsorc/tree/master/src\\
The automated unit tests are located at https://github.com/cjeffery/sworsorc/tree/master/src/test
}

\section[INTRODUCTION]{\bfseries\color{black} INTRODUCTION}

{
\selectlanguage{english}\color{black}
\subsection{HUD View}
Our plan is to integrate the working hud/game code and test that it works in tangent, as opposed to only working in discrete locations of the project. We will be using junit tests to evaluate discrete methods and manual tests to test the GUI and game logic. To plan and document manually test for GUI components as they are developed.  As new code is integrated the unit tests will be rerun to ensure integration did not break any component. Manual tests will be rerun on a case-by-case basis. All tests will be run the day before the end of a sprint and before deliverable presentation.

\selectlanguage{english}\color{black}
\subsection{Rules View} % (fold)
The purpose of this test plan is to state the processes used by Game Rules and Play Team in testing the Spells and Characters for the Swords \& Sorcery project.(Tao, Cameron)
\newline
\newline
This test plan covers the creation and implementation of the Moveableunit class and its sub class Armyunits. (Matt)
\newline
\newline
This test plan is also to provide as much coverage as possible to the methods and data of the Scenario class by verifying the results of execution against expectations through a series of automated unit tests and a manual GUI test. (Tyler)

\subsection
This test plan for the networking portion of the project. It will outline how we test networking, and how we'll test integration of networking into other code.
}


\section[TEST ITEMS]{\bfseries\color{black} TEST ITEMS}

{\selectlanguage{english}\color{black}

The army units will be tested for the proper member variable values as well as proper results from the member function of the Moveableunit and Armyunit classes. Variables will tested at creation and members variable will be tested for proper results when applicable. Some example tests will be that the location member variable is properly set during movement. Also another type of test that will be preformed is that conjured units are properly created and and placed in the proper place on the game board. 
}

{\selectlanguage{english}\color{black}
\subsection{Character}
\begin{itemize}
\item Class Interfaces
\item Class Interactions
\item Spells Implementation
\item Character 
\end{itemize}

\subsection{Scenario}
The data read by the Scenario test will be read from one of the simple 
scenario configuration files and then verified against expectations through 
the class's accessor methods. These data items include 
\newline
\begin{itemize}
\item The scenario's name
\item Number of players
\item Game length
\item The blue sun's initial position
\item The names of armies in the scenario
\item The controlling players of these armies
\item The setup order of these armies
\item The movement order of these armies
\item Names of nations within these armies
\item Names of the neutrals in the scenario
\item The provinces controlled by a nation
\item The characters within a nation
\item The units within a player nation
\item The units within a neutral nation
\item The races of both neutrals and player nations
\item The reinforcement and replacement description strings
\item Data related to where a neutral is leaning toward
\item Whether or not a neutral accepts human sacrifice
\end{itemize}

Unfortunately, the most complex functionality of the Scenario class 
cannot be tested by automated unit tests. The unit pool populator requires 
a manual check.

%% CCC
\subsection{Network}
\begin{itemize}
  \item Client connects and disconnects from the server.
  \item Server detects new client connections.
  \item Client and server can exchange messages.
  \item Client and server are able to identify message type and content.
  \item Client messages are received by related clients.
  \item Network events trigger GUI events properly.
\end{itemize}
%% VVV

\subsection	{Additional items to be tested.}

{\selectlanguage{english}\color{black}
\begin{itemize}
\item GUI Tests
\begin{itemize}
\item Hexes/Units tile ok, look ok
\item Mouse clicks work
\item Menu navigation
\item Game starting (networked, scenario)
\item General gameplay (to the extent it's implemented)
\item Unit movement
\item Other
\end{itemize}
\item Other manual tests (game compiles and runs on different platforms)
\item Auto tests - Junit
\item Ensure all files under resources can be loaded (if feasible)
\item The user can initiate connection and disconnection from the server.
\item The user can... (Network stuff)
\end{itemize}
}

\section[SOFTWARE RISK ISSUES]{\bfseries\color{black} SOFTWARE RISK ISSUES}
{\selectlanguage{english}\color{black}
Software to be tested includes the following:

\begin{enumerate}
\item User's network configuration prevents networking.
\item  GUI - Doesn't load working map, can't support unit movement
\item  JSON Library - Loads scenario incorrectly, if not at all
\item  Incorporating Networking with GUI and game logic may be difficult
\item  Undetected Logic Errors (ULEs)
\item  Misinterpretation of Rules
\item  Connection Failures: For whatever reason, we have no access to a server, or can't connect over user's network
\item  Networking code is unable to connect with other code elements.
\item  Networking code can be integrated, but is too complicated for anyone to work with.
\item Improper casting of units
\item Unit ID's interpreted incorrectly
\item Depends on Java's JSON reader and the programmer's understanding of it
\item Complex data structures such as a map of maps
\item Complex loops to iterate over data structures
\item Poor documentation surrounding some of these iterators
\item Poor documentation in general
\item 3rd party library's. Including json-simple-1.1.1 and controlsfx-8.0.5
\end{enumerate}
}
}

\section[FEATURES TO BE TESTED]{\bfseries\color{black} FEATURES TO BE TESTED}
{\selectlanguage{english}\color{black}
\subsection{Scenario}
From the user's perspective, much of this data reading occurs under the hood. 
In all cases except for populating the unit pool, the Scenario class does not 
manipulate any pieces of the rest of the project. Other components read the 
data from the Scenario class. Therefore, while the solar configuration relies 
on a properly-working Scenario class, this is not apparent to the user because 
solar configuration is also handled by SolarConfig and HUDInitializer classes.
\newline
\newline
Therefore, the only visible component being tested by this plan is the placement 
of units and characters into the correct provinces of the map.
\newline
\newline
This is a listing of what is NOT to be tested from both the Users
viewpoint of what the system does and a configuration
management/version control view. This is not a technical description
of the software, but a USERS view of the functions.
\newline
\newline
All data is verified, but testing the Scenario class alone cannot ensure 
that it reaches the HUD successfully. Integration tests are required outside 
of this plan for information such as solar configuration, move order, setup order, 
game length, and diplomacy.
\newline
\newline
What is not tested is the count of the units on the map that correspond to the scenario loaded. Also the type of unit is not test, this is assumed correct. 



\subsection{Movement}

\begin{itemize}
\item conjured unit appear when casted
\item units are properly represented on HUD
\item movement location is correct on map
\item moral status is properly updated after combat
\end{itemize}

\subsection{Character, Spells}
\begin{itemize}
\item Select character on GUI
\item Select Spell
\item Effects of Partial Spells
\item Manna Costing
\end{itemize}
\subsubsection{Character}
\begin{center}
\begin{tabularx}{\linewidth}{|p{2in}|X|p{2in}|}\hline
\hline
    \textbf{Rule Description}
    &
    \textbf{Test Description}
    &
    \textbf{Expected Result}
\\\hline
    Create new character object with its information.
    &
    createCharacter function is called with 
    specific character name in CharacterMaker.java.
    &
    A new character object is created and returned 
    that contains all of the specified characters information. 
\\\hline 
    Potential spells for selected character displayed.
    &
    Select character in GUI, then click cast spell button on sidebar panel.
    &
    List of spells that can be cast by selected character are displayed.
\\\hline    
\end{tabularx}
\end{center}

\subsubsection{Spells}
\begin{center}
\begin{tabularx}{\linewidth}{|p{2in}|X|p{2in}|}\hline
\hline
    \textbf{Rule Description}
    &
    \textbf{Test Description}
    &
    \textbf{Expected Result}
\\\hline
    Character with Power Level should have a spell book.
    &
    Generate a spell book for the character.
    &
    A frame with a list of spells should be shown on the screen.
\\\hline 
    Show spell description.
    &
    Click on the Spell button.
    &
    A frame will be displayed with all information about the spell.
\\\hline   
    A target need to be selected for most of the spells.
    &
    Click on cast button on the frame of displaying information about the spell. 
    Then select a target by right click the target on the map.
    &
    A target is selected to cast the spell.
\\\hline
\end{tabularx}
\end{center}

}

{\selectlanguage{english}\color{black}
\subsection {Additional Features To Be Test}
\begin{itemize}
\item 1 Complete Turn
\item Movement
\item Teleporting
\item Network 
\item Loading
\item Solar Display
\item Diplomacy Display
\end{itemize}
}

\section[FEATURES NOT TO BE TESTED]{\bfseries\color{black}
	 FEATURES NOT TO BE TESTED}
{\selectlanguage{english}\color{black}
All data is verified, but testing the Scenario class alone cannot ensure 
that it reaches the HUD successfully. Integration tests are required outside 
of this plan for information such as solar configuration, move order, setup order, 
game length, and diplomacy.
\newline
\newline
What is not tested is the count of the units on the map that correspond to the scenario loaded. Also the type of unit is not test, this is assumed correct. 
\begin{itemize}
\item Full game
\item Everything in backlog
\end{itemize}


\subsection{Network}
JUnit tests will be used to test both sides of the network (that is,
the client and the server networking code), in isolation.

Manual testing will be needed to ensure integration of the networking
code with elements of the graphical user interface, as well as testing
that the client and server work over physically distinct machines.
Because of the limited resources for manual testing, manual testing
will test a subset of possible interactions, with the assumption that
core network obstacles will block all messages (i.e. no connection can
be made over the network), or no messages. The graphic effects of
network events will also be tested manually.
}

\section[APPROACH]{\bfseries\color{black} APPROACH}
{\selectlanguage{english}\color{black}
The plan is to use Junit tests and manual tests to ensure that our code follows the rules of the game and works itself. Junit tests are done according to the individuals who have developed the methods being tested, and those are not listed here. However, they should be able to be run together and work. The manual tests will work as though a mock player (tester) is running the game. They should be able to select a unit, move a unit, teleport a unit, advance and view the solar display, send chat messages, and view the diplomacy display. Functionalities of each display should also be tested. For example, a unit with a move allowance should only be viewed moving at or under their limit and a chat message should be sent from one user and be seen by all users.
}

\subsection{Network}
JUnit tests will be used to test both sides of the network (that is,
the client and the server networking code), in isolation.

Manual testing will be needed to ensure integration of the networking
code with elements of the graphical user interface, as well as testing
that the client and server work over physically distinct machines.
Because of the limited resources for manual testing, manual testing
will test a subset of possible interactions, with the assumption that
core network obstacles will block all messages (i.e. no connection can
be made over the network), or no messages. The graphic effects of
network events will also be tested manually.


\section[ITEM PASS/FAIL CRITERIA]{\bfseries\color{black}
	 ITEM PASS/FAIL CRITERIA}
{\selectlanguage{english}\color{black}
All tests should meet the specifications of the Swords and Sorcery board game rules manual, as well as follow logical design patterns and language rules/conventions. The manual tests of movement should fall within 20\% of total movement possibilities the movement rules indicate.
}

\section[SUSPENSION CRITERIA]{\bfseries\color{black}
	 SUSPENSION CRITERIA AND RESUMPTION REQUIREMENTS}
{\selectlanguage{english}\color{black}
\begin{itemize}
\item If the github project is broken or otherwise compromised, productive development and testing would be suspended for a time. To resume, fix whatever has been broken on github.
\item If any component test fails, the larger dependent pieces are unable to be tested until the component is fixed. In such a case, fix the component and continue testing.
\end{itemize}

\section[TEST DELIVERABLES]{\bfseries\color{black} TEST DELIVERABLES}
{\selectlanguage{english}\color{black}
\begin{itemize}
\item Test plan document
\item Test cases
\item Relevant error logs or problem reports
\item Possible solutions
\end{itemize}
}

\section[REMAINING TEST TASKS]{\bfseries\color{black} REMAINING TEST TASKS}
{\selectlanguage{english}\color{black}
There are many remaining test tasks, as we have not achieved a working game yet. Remaining tests include full combat, spell casting, scenario loading, full gameplay, etc. We also anticipate unexpected integration requirements to be added to the test plan.}

\section[ENVIRONMENTAL NEEDS]{\bfseries\color{black} ENVIRONMENTAL NEEDS}
{\selectlanguage{english}\color{black}
In the final test, we'll need a remote machine to host the server code. Along with two computers connected over the internet too the server. Otherwise, we cannot test full networking capabilities and full gameplay as intended. 
}

\section[RESPONSIBILITIES]{\bfseries\color{black} RESPONSIBILITIES}
{\selectlanguage{english}\color{black}
The hierarchy of "inchargeness" is as follows:
\begin{enumerate}
\item Dr. J
\item John Goettsche
\item Everyone else
\end{enumerate}
}
\section[SCHEDULE]{\bfseries\color{black} SCHEDULE}
{\selectlanguage{english}\color{black}
Junit tests should be implemented and run as functionality is added to classes and packages. Manual testing will be done as the GUI is developed and functionality added, as well as when more aspects of other group's code are added. Test to be rerun before demonstration day.
}

\section[PLANNING RISKS AND CONTINGENCIES]{\bfseries\color{black}
	 PLANNING RISKS AND CONTINGENCIES}
{\selectlanguage{english}\color{black}
As the end of the semester is a fixed date, there are no contingencies for failure. Public beatings will be carried out as needed.}

\section[APPROVALS]{\bfseries\color{black} APPROVALS}
{\selectlanguage{english}\color{black}
The only person capable of fully approving any fixes/modifications is Dr. J.}

\section[GLOSSARY]{\bfseries\color{black} GLOSSARY}
{\selectlanguage{english}\color{black}
Junit test is a discrete code test designed to be ran as part of an automated test sequence that tests all Junit tests.}


\clearpage\setcounter{page}{1}\pagestyle{Convertviii}
\section[APPENDIX A. \ [Example JUnit Test{]}]{\selectlanguage{english}\bfseries\color{black} APPENDIX A.
\ Example JUnit Test}


{\selectlanguage{english}\color{black}

\begin{tabbing}
\hspace*{1cm}\=\hspace*{0cm}\= \kill
    @Test\\
    public void testStackWarning() \{\\
        \\
        \>boolean test;\\
        \>pool.addUnit(1, new Bow(), "0101");\\
        \>pool.addUnit(1, new Bow(), "0101");\\
        \\
        \>test = HexStack.overStackWaring(pool.getUnitsInHex("0101"),false);\\
        \\
        \>assertFalse(test);\\
        \\
        \>pool.addUnit(1, new Bow(), "0101");\\
        \\
        \>test = HexStack.overStackWaring(pool.getUnitsInHex("0101"),false);\\
        \>assertTrue(test);\\
    \}\\
    \end{tabbing}  
}

\clearpage\setcounter{page}{1}\pagestyle{Convertviii}
\section[APPENDIX B. \ [Example Manual Test{]}]{\selectlanguage{english}\bfseries\color{black} APPENDIX B.
\ Example Manual Test}
This test will ensure that the graphical components of the over stack warning in the stack class is behaving as designed.\

\begin{enumerate}
\item Start with any hex that has one unit in it.
\item Add one unit to the stack. No warning should be triggered.
\item Add one unit to the stack a warning should be triggered.
\item Close the over stack warning box.
\item Remove one unit from the stack. No warning should be triggered.
\item Remove one unit from the stack. No warning should be triggered.
\item Add one unit to the stack. No warning should be triggered.
\item Add one unit to the stack a warning should be triggered.
\item add one unit to the staff a warning should be triggered indicating there are 2 too many units in the stack.
\end{enumerate}
\
\
The test pass if all steps are successfully completed. Otherwise, test fails.

\clearpage\setcounter{page}{1}\pagestyle{Convertviii}
\section[APPENDIX C. \ [Team Test Plans{]}]{\selectlanguage{english}\bfseries\color{black} APPENDIX C.
\ Individual Team Test Plans}
All individually developed tests to be found at the following link:\\
https://github.com/cjeffery/sworsorc/tree/master/doc/TestPlans
